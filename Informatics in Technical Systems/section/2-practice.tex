\begin{center}
    Задача
\end{center}
Для индивидуально выбранной системы счисления провести следующие операции:
\begin{itemize}
    \item Перевод из десятичной системы в заданную
    \item Перевод из заданной системы в десятичную
    \item Сложение чисел
    \item Вычитание чисел
    \item Умножение чисел
\end{itemize}

Для практических задач данной работы была выбрана система счисления с основанием $7$. Соответственно, обозначения числовых значений в ней будут состоять из символов $0123456$.
\vspace{28pt}

Задание 1. Перевести данное число из десятичной системы счисления в заданну.
\begin{enumerate}
    \item $464_{10}$
    \item $380.1875_{10}$
    \item $115.94_{10}$ (получить пять знаков после запятой в двоичном представлении).
\end{enumerate}

Решение для такой задачи можно реализовать в виде программы на языке python:

\lstinputlisting[language=Python]{program/main.py}

Результаты выполнения программы:
\begin{lstlisting}
Enter number:
380.1875

Your number with base 7:
0.1875
Whole part:  1052
Fractional part:  12121
Result:  1052.12121
\end{lstlisting}

\vspace{28pt}
Задание 2. Перевести данное число в десятичую систему счисления
