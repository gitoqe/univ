\begin{center}
    Теоретическая часть
\end{center}

Системой счисления называется совокупность приемов наименования и записи чисел. В любой системе счисления для представления чисел выбираются некоторые символы (их называют цифрами), а остальные числа получаются в результате каких-либо операций над цифрами данной системы счисления.

Система называется позиционной, если значение каждой цифры (её вес) изменяется в зависимости от её положения (позиции) в последовательности цифр, изображающих число.

Число единиц какого-либо разряда, объединяемых в единицу более старшего разряда, называют основанием позиционной системы счисления. Если количество таких цифр равно $P$, то система счисления называется $P$-ичной. Основание системы счисления совпадает с количеством цифр, используемых для записи чисел в этой системе счисления.

Запись произвольного числа $x$ в $P$-ичной позиционной системе счисления основывается на представлении этого числа в виде многочлена:
\\

$
x = a_n * P^n +
a_{n-1} * P^{n-1} +
... +
a_1 * P^1 +
a_0 * P^0 +
a_{-1} * P^{-1} +
... +
a_{-m} * P^{-m}
$
\\

Арифметические действия над числами в любой позиционной системе счисления производятся по тем же правилам, что и десятичной системе, так как все они основываются на правилах выполнения действий над соответствующими многочленами. При этом нужно только пользоваться теми таблицами сложения и умножения, которые соответствуют данному основанию $P$ системы счисления.

При переводе чисел из десятичной системы счисления в систему с основанием $P > 1$ обычно используют следующий алгоритм:

1) если переводится целая часть числа, то она делится на $P$, после чего запоминается остаток от деления. Полученное частное вновь делится на $P$, остаток запоминается. Процедура продолжается до тех пор, пока частное не станет равным нулю. Остатки от деления на $P$ выписываются в порядке, обратном их получению;

2) если переводится дробная часть числа, то она умножается на $P$, после чего целая часть запоминается и отбрасывается. Вновь полученная дробная часть умножается на $P$ и т.д. Процедура продолжается до тех пор, пока дробная часть не станет равной нулю. Целые части выписываются после запятой в порядке их получения. Результатом может быть либо конечная, либо периодическая дробь в системе счисления с основанием $P$. Поэтому, когда дробь является периодической, приходится обрывать умножение на каком-либо шаге и довольствоваться приближенной записью исходного числа в системе с основанием $P$.
\clearpage
